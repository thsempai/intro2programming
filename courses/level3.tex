% Level 3:
% Matière:
% ● Boucle logique (while)
% ● Comprendre l’utilité des fonction (et des méthodes)
% ● import de \emph{module} Python
% ● utilisation de nombres aléatoires
% Exercices:
% 1. Remplacer la boucle for du bonus stage du level 2 par une boucle while
% conditioner sur la victoire (arrivée à la fin du labyrinthe) du héro.
% 2. Coder le mouvement des ennemis pour qu’il bouge de manière aléatoire, tout en
% évitant les obstacles.
% Bonus Stage:
% Faire sortir le héro du du donjon en évitant les obstacles et en attendant que les
% ennemis

% horaire du cours:

% 8h45
% 10h30
% 12h15
% 13h15
% 15h00
% 16h45


% présence: 10 min.
% -- 8h55
% Révision:
% 	Exercice: 30 min.
%	Correction: 20 min.  
% -- 9h45
% Boucle while 5 min.
% Exercice 10 min.
% -- 10h00
% Import module personnelle 5 min + Exercice : 5 min.
% Import module Python: 5 min.
% Excercice Random : 5 min.
% -- 10h15
% Explication du random: 5 min
% Exercices sur le random: 10 min
% PAUSE

\chapter{Level 3}

\section{Les boucles logiques}

On a vu, dans le chapitre précédent, la boucle dite arithmétique. Avec cette boucle, la partie de code \emph{contrôlée} par celle-ci s'exécute un nombre de fois précis.

Parfois, on sera intéressé par faire en sorte que le programme s'arrête non pas après un certain nombre de fois, mais plutôt lorsqu'une certaine condition est remplie. Pour ce faire, on fait appel aux \emph{boucles logiques}.

En Python, c'est l'instruction \codeintext{while}\footnote{"Tant que" en anglais} qui est utilisée.

Cette instruction doit être suivie d'une \emph{condition}, celle-ci va déterminer à chaque \emph{itération}, si le programme doit continuer ou non d'exécuter la boucle.

Dans le script suivant, la boucle sera exécutée tant qu'il reste des monstres.

\begin{lstlisting}
while monsters_nbr >= 0:
	hero_kills = hero.attacks()
	monster_nbr = monster_nbr - hero_kills
	game.run()
\end{lstlisting}

La méthode \codeintext{attacks} de la variable \codeintext{hero} renvoie 1 ou 0, donc il est impossible de savoir combien d'itérations seront nécessaires pour réduire la variable \codeintext{monsters\_nbr} à 0.
C'est donc un \emph{boucle logique} qui est utilisée dans ce genre de cas.

%% voir dans le level 2 comment Bastien parle  des booleens.
La condition qui contrôle la boucle est similaire à la condition utilisée avec l'instruction \codeintext{if}.
La boucle s'arrête lorsque la condition est \emph{faux}.

Dans le cas de notre script, c'est lorsque le nombre de monstres passe en-dessus de 0 que la condition devient \emph{fausse}.

\subsection{\`A retenir}

\textbf{Très, très, très important}: il est impératif qu'à l'intérieur de votre \emph{boucle logique}, il y ait un moyen de rendre la condition de continuation fausse. Sinon vous allez vous retrouver face à une boucle infinie. 

\subsection{Stage 3-1}

Reprenez le \textit{bonus stage} du \textit{level 2}\footnote{au besoin le code corrigé de cet exercice se trouve dans le script \path{bonus\_stage\_2.py}}. Arrangez ce code avec une \emph{boucle logique} à la place de la \emph{boucle arithmétique} pour que le héros arrête de bouger dès qu'il est arrivé à destination et non après un nombre d'itération arbitraire.
Lorsque le héros a atteint sa cible, la méthode \codeintext{hero.is\_at\_goal()} renvoie \codeintext{True}, dans les autres cas elle renvoie \codeintext{False}.

\section{imports}

En Python, il est courant de devoir \emph{importer} le contenu d'un fichier dans un autre fichier.
\`A plusieurs reprises, on déjà vu cette instruction dans les scripts que l'on a utilisés:

\begin{lstlisting}
import world
\end{lstlisting}

Cette commande, demande à Python d'\emph{importer} le contenu du fichier \path{world.py} en vue de l'utiliser plus tard.
Dans la suite, si l'on veut faire appel au contenu de \codeintext{world}, on doit utiliser son nom suivit d'un "\codeintext{.}"

Par exemple, si je veux utiliser la variable \codeintext{GROUND\_IMAGE\_PATH} dans mon script, je vais procéder comme suit:

\begin{lstlisting}
import world

scree_size = world.GROUND_IMAGE_PATH
\end{lstlisting}

On peut aussi n'\emph{importer} qu'une partie d'un \emph{module}\footnote{C'est le nom que l'on donne au fichier que l'on importe}.

\begin{lstlisting}
from world import GROUND_IMAGE_PATH
\end{lstlisting}

En procédant de cette manière, on n'\emph{importe} que la variable \codeintext{GROUND\_IMAGE\_PATH}.
Cela présente l'avantage de ne pas devoir précéder la variable \codeintext{GROUND\_IMAGE\_PATH} de \codeintext{world.}, on peut se contenter du nom de la variable seul.

\begin{lstlisting}
from world import GROUND_IMAGE_PATH

screen_size = GROUND_IMAGE_PATH
\end{lstlisting}

Et enfin, on peut \emph{importer} plusieurs composants d'un \emph{module} sur la même ligne.
Si je voulais importer \codeintext{GROUND\_IMAGE\_PATH} et la variable \codeintext{OBSTACLE\_IMAGE\_PATH}, je procéderait ainsi:

\begin{lstlisting}
from world import GROUND_IMAGE_PATH, OBSTACLE_IMAGE_PATH
\end{lstlisting}

\subsection{Stage 3-2}
Dans le fichier \path{data.py} se trouvent toutes les données utilisées pour le jeu. Parmis ces données se trouve la donnée de la \textit{taille de la fenêtre du jeu} : \codeintext{SCREEN\_SIZE} (c'est un \codeintext{tuple} correspondant à \codeintext{(\textit{largeur},\textit{hauteur})} et la donnée du \textit{facteur de grossissement}: \codeintext{SCALE\_FACTOR}.
La grandeur réelle de l'écran est calculée en multipliant la \textit{largeur} et la \textit{hauteur} de l'écran par le \textit{facteur de grossissement}.
\emph{Importez} ces données dans le \codeintext{run\_game} et affichez dans la console la taille réelle de la fenêtre de jeu.

% B : Hahaha excellent XD
\subsection{N'Import quoi!}

Il n'y a pas que les \emph{modules} que vous avez faits vous-même que vous pouvez \emph{importer}. Python est livré avec un série de \emph{modules} prêts à être utilisés par vos mains avides de coder.
Pour importer ces \emph{modules}, il n'y a pas de différences par rapport à pour vos \emph{modules} personnels. Il suffit juste de connaitre le nom du \emph{module} que vous voulez importer\footnote{La liste des modules disponibles dans Python et leur contenu est disponible sur le site de Python (\url{www.python.org/doc})}. L'un de ces modules qui nous sera très utile dans le cadre de la réalisation de jeux vidéos, c'est \codeintext{random}.

Pour l'importer donc pas de mystère:

\begin{lstlisting}
import random
\end{lstlisting}

\subsection{Stage 3-3}
Ce module contient la fonction \codeintext{(randint)}. Cette fonction prend deux arguments qui doivent être de type \codeintext{int}.
En important cette fonction et uniquement cette fonction (Rappelez-vous, les méthodes d'\emph{import} de \emph{modules} Python ne sont pas différentes de l'\emph{import} de modules personnels), essayez de déterminez ce que fait la fonction (même si le nom du \emph{module} et de la fonction sont de gros indices).
Vous pouvez utiliser le script \codeintext{run\_game}, ou bien la console Python pour faire ces tests.

\section{La génération de nombres aléatoires}

Vous avez pû le constater, la fonction \codeintext{randint} génère des nombres aléatoires compris entre le premier argument de la fonction et le deuxième argument\footnote{Pour autant que les arguments soient en ordre croissant, sinon ça génère une erreur.}
Générer des nombres aléatoires est extrêmement pratique, car cela permet de simuler le caractère hasardeux d'une situation, un peu à l'image d'un dé.

\subsection{Stage 3-4}

Utilisez la fonction random sur le jeu.
Simulez un dé à 6 faces:
\begin{description}
	\item[1]: le héros tourne à droite
	\item[2]: le héros tourne à gauche
	\item[3-5]: le héros avance
	\item[6]: le héros arrête de bouger définitivement
\end{description}

En plus de cela, à chaque lancer de dé: affichez dans la console le résultat de celui-ci.


