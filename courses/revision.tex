\documentclass[12pt,a4paper]{article}

\usepackage[francais]{babel}
\usepackage[utf8]{inputenc}
\usepackage{epigraph}
\usepackage[T1]{fontenc}
\usepackage{listings}
\usepackage{graphicx}
\usepackage{parskip}
\usepackage{xcolor}


%% config
\definecolor{deepblue}{rgb}{0,0,0.5}

\graphicspath{{images/}}
\lstset{
    language={Python},
    basicstyle=\ttfamily\small, 
    tabsize=4,
    keywordstyle=\bold,
    commentstyle=\color{gray},
    backgroundcolor=\color{lightgray},
    otherkeywords={self},
	keywordstyle=\ttfamily\small\color{deepblue},
    frame=single
    showtabs=false,
    showspaces=false,
    showstringspaces=false,
    inputencoding=utf8,
    literate={à}{{\`a}}1 {è}{{\`e}}1,
}

%%macro

\newcommand{\path}[1]{\texttt{#1}}

\newcommand{\codeintext}[1]{\texttt{#1}}
\newcommand{\response}{réponse : \hrulefill\\\\}

\begin{document}

\section*{Révisions}

% Pour chaques exercices, donnez ce qui sera inscrit pas la commande \codeintext{print}.

\begin{enumerate}

\item
\begin{lstlisting}
a = 1
b = 2
c = a + b
print(c)
\end{lstlisting}
\response % 3

\item
\begin{lstlisting}
a = 10
a = 12
print(a)
\end{lstlisting}
\response % 12

\item
\begin{lstlisting}
a = 3
b = 2 * a
print(b)
\end{lstlisting} 
\response % 6

\item
\begin{lstlisting}
a = 2
b = 5
c = a + c
print(b + c)
\end{lstlisting}
\response % 12

\item
\begin{lstlisting}
a = min(2, 5)
print(a)
\end{lstlisting}
\response % 2
\pagebreak

\item
\begin{lstlisting}
a = min(2, 5)
b = max(12, 5)
print(a + b)
\end{lstlisting}
\response % 14

\item
\begin{lstlisting}
a = min(12, 5)
b = max(a, 1)
print(b)
\end{lstlisting}
\response % 5

\item
\begin{lstlisting}
a = min(10, 12)
b = max(5, a)
print(a + b)
\end{lstlisting}
\response % 20

\item
\begin{lstlisting}
a = "Luke,"
b = " je suis ton père."
print(a + b)
\end{lstlisting}
\response % "Luke, je suis ton père."

\item
\begin{lstlisting}
a = "The answer : "
b = 42
print(a + str(b))
\end{lstlisting}
\response % "The answer: 42"
\pagebreak

\item
\begin{lstlisting}
a = "12"
b = int(a) + 5
print(a + b)
\end{lstlisting}
\response % 17

\item
\begin{lstlisting}
a = 3
a = str(a) * 2
print("Dites " + a)
\end{lstlisting}
\response % "Dites 33"

\item
\begin{lstlisting}
a = 11
a = a // 4
print(a)
\end{lstlisting}
\response % 2

\item
\begin{lstlisting}
a = range(3) 
print(list(a))
\end{lstlisting}
\response % [0, 1, 2]

\item
\begin{lstlisting}
a = [0, 2, 4]
a = len(a)
print(a) 
\end{lstlisting}
\response % 3
\pagebreak

\item
\begin{lstlisting}
a = [0, 2, 4, 2, -1, 3]
a = str(len(a))
a = a * 3
print(a) 
\end{lstlisting}
\response % 666

\item
\begin{lstlisting}
a = [0, 2, 4, 2, -1, 3]
b = a[1]
a = a[2] 
print(a + b) 
\end{lstlisting}
\response % 6

\item
\begin{lstlisting}
a = "Dark Vador"
b = "Lu"
c = b + a[3] + 'e' 
print(c) 
\end{lstlisting}
\response % "Luke"

\item
\begin{lstlisting}
a = [1, 2, 2, 4, 6]
b = a[1:3]
print(b) 
\end{lstlisting}
\response % [2, 2]

\item
\begin{lstlisting}
a = [1, 2, 2, 4, 6]
b = a[2:]
print(b) 
\end{lstlisting}
\response %  [2, 4, 6]
\pagebreak

\item
\begin{lstlisting}
item_nbr = 2
item_name = "patate"
if nbr > 1:
	item_name = item_name + 's'
print(item_name)

\end{lstlisting}
\response %  "patates"

\item
\begin{lstlisting}
item_nbr = 1
item_name = "patate"
if nbr > 1:
	item_name = item_name + 's'
print(item_name)

\end{lstlisting}
\response %  "patate"

\item
\begin{lstlisting}
a = ""

for n in range(3):
	a = a + "B"
print(a)
\end{lstlisting}
\response % "BBB"

\item
\begin{lstlisting}
a = 1

for n in range(3):
	a = a + a

print(a)
\end{lstlisting}
\response % 6

\pagebreak
\item
\begin{lstlisting}
a = ""

for n in range(4):
	a = a + str(n)
	
print(a)
\end{lstlisting}
\response % "0123"

\item
\begin{lstlisting}
a = ""

for n in range(5):
	if n > 1 and n != 3: 
		a = a + str(n)
	
print(a)
\end{lstlisting}
\response % "24"

\end{enumerate}

\end{document}
