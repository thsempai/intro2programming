\PassOptionsToPackage{table}{xcolor}
\documentclass[10pt,a4paper,landscape]{article}
\usepackage{multicol}
\usepackage{calc}
\usepackage{ifthen}
\usepackage[landscape]{geometry}
\usepackage{titlesec}
\usepackage{graphicx}
\usepackage{ stmaryrd }
\usepackage{tikz}
\usetikzlibrary{shadows, calc}
\usepackage{listings}
\usepackage{xcolor}
\usepackage{hyperref}


%% config
\definecolor{deepblue}{rgb}{0,0,0.5}

%\usepackage[table]{xcolor}

% Sublime Text 2 cheat sheet

% This sheet is meant to be compiled with pdflatex.

% This code is based on the LaTeX cheat sheet found here :
% http://www.stdout.org/~winston/latex/
% by Winston Chang

% Licensed under Creative Commons Attribution-NonCommercial-ShareAlike 3.0 Unported License.

% Setting the margins at 1cm everywhere
\geometry{top=1cm,left=1cm,right=1cm,bottom=1cm} 

% Turn off header and footer
\pagestyle{empty}
 
\makeatletter

\newcommand{\vcenteredinclude}[2]{\begingroup
\setbox0=\hbox{\includegraphics[#2]{#1}}%
\parbox{\wd0}{\box0}\endgroup}

\newcommand*\keystroke[1]{%
  \tikz[baseline=(key.base)]
    \node[%
      draw,
      fill=white,
      drop shadow={shadow xshift=0.25ex,shadow yshift=-0.25ex,fill=black,opacity=0.75},
      rectangle,
      rounded corners=2pt,
      inner sep=1pt,
      line width=0.5pt,
      font=\scriptsize\sffamily
    ](key) {~#1~\strut}
  ;
}

% Don't print section numbers
\setcounter{secnumdepth}{0}


\setlength{\parindent}{0pt}
\setlength{\parskip}{0pt plus 0.5ex}

\renewcommand{\familydefault}{\sfdefault}

\titleformat{\section}[block]%              
    {\tikz[overlay] \shade[left color=gray!10,right color=gray] (0,-1ex) rectangle (\linewidth,1.5em); \Large}%    
    {\thesection}%                   
    {1em}%
    {}

\titleformat{\subsection}[block]%              
    {\large}%    
    {\thesubsection}%                   
    {1em}%
    {}

\graphicspath{{images/}}
\lstset{
	literate=
  {á}{{\'a}}1 {é}{{\'e}}1 {í}{{\'i}}1 {ó}{{\'o}}1 {ú}{{\'u}}1
  {Á}{{\'A}}1 {É}{{\'E}}1 {Í}{{\'I}}1 {Ó}{{\'O}}1 {Ú}{{\'U}}1
  {à}{{\`a}}1 {è}{{\`e}}1 {ì}{{\`i}}1 {ò}{{\`o}}1 {ù}{{\`u}}1
  {À}{{\`A}}1 {È}{{\'E}}1 {Ì}{{\`I}}1 {Ò}{{\`O}}1 {Ù}{{\`U}}1
  {ä}{{\"a}}1 {ë}{{\"e}}1 {ï}{{\"i}}1 {ö}{{\"o}}1 {ü}{{\"u}}1
  {Ä}{{\"A}}1 {Ë}{{\"E}}1 {Ï}{{\"I}}1 {Ö}{{\"O}}1 {Ü}{{\"U}}1
  {â}{{\^a}}1 {ê}{{\^e}}1 {î}{{\^i}}1 {ô}{{\^o}}1 {û}{{\^u}}1
  {Â}{{\^A}}1 {Ê}{{\^E}}1 {Î}{{\^I}}1 {Ô}{{\^O}}1 {Û}{{\^U}}1
  {œ}{{\oe}}1 {Œ}{{\OE}}1 {æ}{{\ae}}1 {Æ}{{\AE}}1 {ß}{{\ss}}1
  {ű}{{\H{u}}}1 {Ű}{{\H{U}}}1 {ő}{{\H{o}}}1 {Ő}{{\H{O}}}1
  {ç}{{\c c}}1 {Ç}{{\c C}}1 {ø}{{\o}}1 {å}{{\r a}}1 {Å}{{\r A}}1
  {€}{{\EUR}}1 {£}{{\pounds}}1,
    language={Python},
    basicstyle=\ttfamily\small, 
    tabsize=4,
    keywordstyle=\bold,
    commentstyle=\color{gray},
    otherkeywords={self},
	keywordstyle=\ttfamily\small\color{deepblue},
    showtabs=false,
    showspaces=false,
    showstringspaces=false
}
% -----------------------------------------------------------------------



\begin{document}

%\renewcommand{\familydefault}{\sfdefault}

\newcommand{\ret}{\keystroke{$\hookleftarrow$}}
\newcommand{\shift}{\keystroke{$\Uparrow~$}}
\newcommand{\alt}{\keystroke{Alt}}
\newcommand{\up}{\keystroke{$\uparrow$}}
\newcommand{\down}{\keystroke{$\downarrow$}}
\newcommand{\bkspc}{\keystroke{$\longmapsfrom$}}
\newcommand{\ctrl}[1]{\texttt{\keystroke{Ctrl}#1}}
\newcommand{\codeintext}[1]{\texttt{#1}}
\raggedright



\begin{multicols}{2}

\rowcolors{1}{gray!20}{white}
\begin{LARGE}

Cheat sheet pour Python 3.
\end{LARGE}


\section{Types de variables}
\begin{tabular}{p{3cm}p{4cm}p{\linewidth - 8.25cm}}
\codeintext{int} & Entier & 1 2 3 4 42 \\
\codeintext{float} & R\'eel & 1.2 3.4 5. 42.0 \\
\codeintext{str} & Cha\^ine de caract\`eres & "Texte" 'autre texte'\\
\codeintext{bool} & Bool\'een & True False \\
\codeintext{tuple} & N-uple & (1, 2) (1, 'a', 1.2)\\
\codeintext{list} & Liste & [1, 2, 5] [1, 5, "e"]\\
\end{tabular}

\section{Op\'erateurs math\'ematiques}
\begin{tabular}{p{3cm}p{4cm}p{\linewidth - 8.25cm}}
\codeintext{+} & Addition & 1 + 2\\
\codeintext{+} & Concat\'enation & "Hello" + "World"\\
\codeintext{-} & Soustraction & 3 - 4\\
\codeintext{*} & Multiplication & 5 * 6\\
\codeintext{*} & Multiplication de \codeintext{str} & "6" * 3\\
\codeintext{/} & Division & 7 / 8\\
\codeintext{//} & Division enti\`ere & 9 / 10\\
\end{tabular}

\section{Opérateurs de comparaison}
\begin{tabular}{p{3cm}p{4cm}p{\linewidth - 8.25cm}}
\codeintext{==} & \'Egal & 1 == 2\\
\codeintext{!=} & Diff\'erent & 3 != 4\\
\codeintext{<} & Plus petit & 5 \codeintext{<} 6\\
\codeintext{<=} & Plus petit ou \'egal & 7 \codeintext{<=} 8\\
\codeintext{>} & Plus grand & 9 \codeintext{>} 10\\
\codeintext{>=} & Plus grand ou \'egal & 11 \codeintext{>=} 12\\
\codeintext{in} & Est dedans \'egal & 7 in ([1, 2, 3])\\
\end{tabular}


\section{Fonctions Python}
\begin{tabular}{p{3cm}p{4cm}p{\linewidth - 8.25cm}}
\codeintext{print} & Affiche un texte dans la console & \codeintext{print("Hello!")}\\
\codeintext{input} & Renvoie la valeur entr\'ee par l'utilisateur sous forme de \codeintext{str}& \codeintext{name = input("Name: ")}\\
\codeintext{len} & Renvoie la longueur d'une \codeintext{list}, d'un \codeintext{tuple} ou d'une \codeintext{str} pass\'e en param\^etre.  & \codeintext{len{[1, 2, 3]}}\\
\end{tabular}

\section{M\'ethodes de \codeintext{list}}
\begin{tabular}{p{3cm}p{4cm}p{\linewidth - 8.25cm}}
\codeintext{append} & Ajoute une valeur en fin de \codeintext{list}. & \codeintext{ma\_list.append(3)}\\
\codeintext{insert} & Ins\`ere une valeur à l'index pass\'e en param\^etre. \textit{insert(index, valeur)} & \codeintext{ma\_list.insert(0, 'test')}\\
\codeintext{remove} & Retire la premi\`ere occurance de la valeur pass\'ee en param\^etre. & \codeintext{ma\_list.remove('t')}\\
\codeintext{pop} & Retire la valeur à l'index pass\'e en param\^etre. Et renvoie cette valeur. & \codeintext{first = ma\_list.pop(0)}\\
\codeintext{sort} & Trie la \codeintext{list} par ordre croissant. & \codeintext{ma\_list.sort()}\\
\codeintext{reverse} & Inverse le contenu de la \codeintext{list}. & \codeintext{ma\_list.reverse()}\\
\end{tabular}

\section{Random}
\begin{tabular}{p{3cm}p{4cm}p{\linewidth - 8.25cm}}
\codeintext{randint} & Renvoie un nombre al\'eatoire compris entre les deux \codeintext{int} pass\'es en param\^etre. & \codeintext{bingo = randint(1, 50)} 
\end{tabular}

\section{Import}

Pour importer un module:
\begin{lstlisting}
import <nom du module>
\end{lstlisting}

Pour importer une partie du module:
\begin{lstlisting}
from <nom du module> import <partie du module>
\end{lstlisting}

Exemple:

\begin{lstlisting}
import random
from math import sin
\end{lstlisting}

\section{Boucle arithm\'etique}

\begin{lstlisting}
for <variable> in <list, str ou range()>:
	<code contrôlé par la boucle>
\end{lstlisting}


Exemple:
\begin{lstlisting}
number = 2
ma_list = []

for n in range(3):
	number = number + number
	lma_listappend(a)
	
for value in ma_list:
	print(value)
\end{lstlisting}

\section{Condition}

\begin{lstlisting}
if <condition>:
	<code contrôlé par la condition>
elif <autre condition>:
	<code contrôlé par l'autre condition>
else:
	<code contrôlé éxécuté si les autres coditions n'ont pas été remplies>	
\end{lstlisting}

Il peut y avoir autant de \codeintext{elif} que d\'esir\'e, par contre il ne peut y avoir qu'un \codeintext{else} qui est toujours mis à la fin du bloc de condition.

Exemple:

\begin{lstlisting}
import random

response = randint(1, 6)
test = int(input("tentative: ")
if response == tentative:
	print("Well done!")
elif response == 42:
	print("Cà c'est LA REPONSE")
else:
	print("Dommage")
\end{lstlisting}


\section{Boucle logique}

\begin{lstlisting}
while <condition>:
	<code contrôlé par la boucle>
\end{lstlisting}


{\normalsize exemple:}
\begin{lstlisting}
result = []
 
while len(result) < 10:
	response = input("Donne une valeur: " )
	if response not in result:	
		result.append(response)
\end{lstlisting}

\end{multicols}
\end{document}
