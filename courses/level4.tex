\chapter{Level 4}

\section{Les fonctions}

Une \emph{fonction} est un outil qui permet d'abstraire le code afin de garder un maximum de structure et de lisibilité dans le code.

En gros, une \emph{fonction} peut être vue comme un bout de code sur lequel on aurait mis un nom.

Nous avons déjà utilisé plusieurs fonctions : \codeintext{print}, \codeintext{input}, \codeintext{len}, ...

Nous avons même déjà vu, sans savoir exactement ce que c'était, comment définir une \emph{fonction} dans le code de notre jeu:


\begin{lstlisting}
import world


def main():
    dungeon_size = (10, 10)
    game = world.Game(dungeon_size)
    hero = game.hero
\end{lstlisting}

Dans ce morceau de code la \emph{fonction} se nomme \codeintext{main}.

Pour définir une fonction, il faut simplement écrire \codeintext{def} suivit de parenthèses\footnote{Nous verrons dans la suite du cours que ces parenthèses ne seront pas toujours vides.} et d'un "\codeintext{:}
Le code de la fonction, comme pour les boucles ou la condition, doit être indenté.

Pour appeler une fonction, il suffit d'utiliser son nom, comme suit:

\begin{lstlisting}
main()
\end{lstlisting}

Dans ce cas, c'est la fonction \codeintext{main} qui est appelée.

\subsection{Important}
Pour la clareté de votre code et pour faciliter sa relecture, il est très important que le nom de votre fonction soit explicite.
Ne nommez pas une variable qui simule le lancement d'un dé \codeintext{a} ou même \codeintext{d}, mais plutôt \codeintext{dice} ou \codeintext{dice\_simulation}

\subsection{Stage 4-1}

Dans le jeu du bingo (Stage 3-4), ajoutez une \emph{fonction} \codeintext{bingo} qui affiche "Bingo!" dans la console.
Dans le code principal du jeu, remplacez la partie qui s'occupe d'écrire "Bingo!" par l'appel de votre fonction

\section{Le retour}

L'une des caractéristiques d'une \emph{fonction} c'est qu'elle peut avoir un \emph{retour}.
Par exemple, le nombre aléatoire donné par la fonction \codeintext{randint} est son \emph{retour}.

Un \emph{retour} est une valeur donnée par une \emph{fonction}. On dit d'ailleurs que la fonction \emph{retourne} une valeur.

Lorsqu'il y a besoin de retourner une valeur on utilise le mot clé \codeintext{return}.

Le \emph{retour} d'une fonction doit bien évidemment être recueilli par le code appelant, sinon il est perdu.

\begin{lstlisting}
import random


def dice_simulation():
	dice = random.randint(1, 6)
	return dice
	
result = dice()
print("Résultat du dé: " + result)	
\end{lstlisting}

\section{Stage 4-2}
Faites en sorte de bouger dans une \emph{fonction} la partie du code qui demande à l'utilisateur de choisir un chiffre et faites en sorte que cette \emph{fonction} \emph{renvoie} le nombre choisi par l'utilisateur.

\section{Paramètres}

Une fonction peut aussi avoir un ou plusieurs \emph{arguments}. Les \emph{arguments} sont des valeurs qui sont transférées du code appelant à la \emph{fonction} via des variables appellées \emph{paramètres}

Dans le code suivant:

\begin{lstlisting}
from random import randint


dice = randint(1, 6)
\end{lstlisting}

1 et 6 sont les \emph{paramètres} de \codeintext{randint}.

On définit les paramètres d'une fonction en donnant des noms de variables (qui seront ensuite utilisés dans le code de la fonction).

\begin{lstlisting}
def double(number):
	return number * 2
\end{lstlisting}

Dans cet exemple un peu bateau, \codeintext{number} est l'argument de la \emph{fonction} \codeintext{double}.

\begin{lstlisting}
def list_multiplication(array, number):
	clone = []
	
	for n in array:
		clone.append(n * number)

	return clone
\end{lstlisting}

Dans cet autre exemple, \codeintext{array} et \codeintext{number} sont les arguments de la fonction \codeintext{list\_multiplication}

\subsection{Stage 4-3}

Dans le bingo, faites en sorte de sortir le code qui vérifie le nombre rentré par le joueur dans une \emph{fonction}.
Le code affichera si le nombre est plus petit ou plus grand, et renverra \codeintext{True} si le joueur à deviné juste et \codeintext{False} dans les autre cas.
La \emph{fonction} prendra en \emph{paramètres} le nombre donné par le joueur ainsi que le nombre à deviner.

\section{Valeur par défaut}

Il y a moyen de donner une \emph{valeur par défaut} à un argument.
Cette valeur lui sera donnée si le code appelant ne le fait pas.

\begin{lstlisting}
import random


def dice_simulation(faces=6):
	dice = random.randint(1, faces)
	return dice
\end{lstlisting}

Dans le script ci-dessus, l'\emph{argument} a une \emph{valeur par défaut} de 6.
Si le code appelant est:
\begin{lstlisting}
result = dice_simulation(10)
\end{lstlisting}

le \emph{paramètre} \codeintext{faces} vaut 10, mais si on ne l'avait pas précisé, comme dans le code suivant:

\begin{lstlisting}
result = dice_simulation()
\end{lstlisting}

le \emph{paramètre} \codeintext{faces} vaut 6 dans ce cas là.

Il est possible de mélanger des \emph{arguments} sans \emph{valeur par défaut} et des argument avec \emph{valeur par défaut} dans une fonction, mais il faut impérativement que les \emph{paramètres} avec \emph{valeur par défaut} soient en dernier dans la liste d'\emph{arguments}.

\begin{lstlisting}
import random


def dices_simulation(dice_number, faces=6):
	result = 0
	
	for n in range(dice_number)
		dice = random.randint(1, faces)
		result = result + dice
	return result
\end{lstlisting}

Dans le code du bingo, modifiez la détermination des bornes du jeu à l'aide d'une \emph{fonction} avec une \emph{valeur par défaut}.
La première fois, le nombre est choisi de manière normale, mais à chaque fois que le joueur recommence de jouer, la borne maximum est modifiée de manière à valoir \codeintext{born\_max + (7 - tentatives}.
La toute première fois où elle est appellée, la \emph{fonction} n'aura pas de \emph{paramètre}, et les autres fois elle aura le nombre de tentatives comme \emph{paramètre}.