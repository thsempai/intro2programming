\chapter{Conclusion}

Et voilà, nous sommes arrivés à la fin de ce premier cours de programmation avec Python.

L'objectif du cours était double : d'une part vous faire prendre goût à la programmation, et d'autre part vous donner les bases et les clefs qui vous permettront d'aborder des cours plus ardus dans le domaine. Si au moins vous n'êtes pas dégoûtées de Python et de la programmation en général, mission accomplie !

Ceci étant dit, il va de soi que le présent cours n'est qu'une première étape dans le long trajet qui vous mènera à une véritable maîtrise du sujet. Mais nous espérons en tous cas vous avoir montré que ce chemin peut être agréable, et même, surtout dans le cadre du monde du jeu vidéo, fun.

Vous serez étonnées de voir à quel point les principes que vous avez vu pendant ces quelques jours, les \codeintext{if}, \codeintext{for}, \codeintext{while} et autres variables, sont véritablement omniprésents, que vous fassiez du RPG Maker, du Unity3D ou de l'Unreal Engine.

Si vous trouvez des incohérences dans ces notes de cours, ou qu'elles vous semblent pouvoir bénéficier de clarifications ou autres, n'hésitez pas à nous faire parvenir vos remarques, nous serons ravis de les prendre en compte !

Game on !

---Thomas Stassin et Bastien Gorissen