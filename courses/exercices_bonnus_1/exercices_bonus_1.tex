\documentclass[12pt,a4paper]{article}

\usepackage[francais]{babel}
\usepackage[utf8]{inputenc}
\usepackage{epigraph}
\usepackage[T1]{fontenc}
\usepackage{listings}
\usepackage{graphicx}
\usepackage{parskip}
\usepackage{xcolor}


%% config
\definecolor{deepblue}{rgb}{0,0,0.5}

\graphicspath{{images/}}
\lstset{
    language={Python},
    basicstyle=\ttfamily\small, 
    tabsize=4,
    keywordstyle=\bold,
    commentstyle=\color{gray},
    backgroundcolor=\color{lightgray},
    otherkeywords={self},
	keywordstyle=\ttfamily\small\color{deepblue},
    frame=single
    showtabs=false,
    showspaces=false,
    showstringspaces=false,
    inputencoding=utf8,
    literate={à}{{\`a}}1 {è}{{\`e}}1,
}

%%macro

\newcommand{\path}[1]{\texttt{#1}}

\newcommand{\codeintext}[1]{\texttt{#1}}
\newcommand{\response}{réponse : \hrulefill\\\\}

\begin{document}

\section*{Exercices bonus 1}

Dans les scripts suivants, trouvez l'erreur qui s'est glissée dans le code et une fois celle-ci corrigée, déterminez ce qui sera affiché à l'écran.

Vous pouvez vous aidez des scripts joints à ce fichier, pour trouver l'erreur et la corriger. \\

\begin{enumerate}

\item
\begin{lstlisting}
a = 1
b = 2

if a + b > 2
	c = a + b
else:
	c = 2
print(c)
\end{lstlisting}
\response % 3

\item
\begin{lstlisting}
a = 3

for x in range(a):
	a = a + x1
print(a + a)   
\end{lstlisting}
\response % 12

\begin{lstlisting}
name = 'Harry'
index = 0

while index < len(name):
	if name[index] = 'a':
		name = name[:index] + '4' + name[index+1:]
	elif name[index] == 'i':
		name = name[:index] + '1' + name[index+1:]

	index = index + 1

print(name)
\end{lstlisting}
\response % H4rry

\begin{lstlisting}
def p(n, m=1):
	r = 1
	
	for i in range(m):
	r = r * n

	return r

print(p(4,2) + p(3,3) + p(2)) 
\end{lstlisting}
\response % 45

\begin{lstlisting}
def mix(m1, m2):
	m3 = ""
	for index in range(len(m1)):
		if index < len(m2):
			m3 = m3 + m1[index] + m2(index)
	return m3
	
a = "Al ean rcan"
b = " asmiepohie"
print(mix(a, b)) 	

\end{lstlisting}
\response % A la semaine prochaine

\end{enumerate}

\end{document}
