\documentclass[12pt,a4paper]{article}

\usepackage[francais]{babel}
\usepackage[utf8]{inputenc}
\usepackage{epigraph}
\usepackage[T1]{fontenc}
\usepackage{listings}
\usepackage{graphicx}
\usepackage{parskip}
\usepackage{xcolor}


%% config
\definecolor{deepblue}{rgb}{0,0,0.5}

\graphicspath{{images/}}
\lstset{
    language={Python},
    basicstyle=\ttfamily\small, 
    tabsize=4,
    keywordstyle=\bold,
    commentstyle=\color{gray},
    backgroundcolor=\color{lightgray},
    otherkeywords={self},
	keywordstyle=\ttfamily\small\color{deepblue},
    frame=single
    showtabs=false,
    showspaces=false,
    showstringspaces=false,
    inputencoding=utf8,
    literate={à}{{\`a}}1 {è}{{\`e}}1,
}

%%macro

\newcommand{\path}[1]{\texttt{#1}}

\newcommand{\codeintext}[1]{\texttt{#1}}
\newcommand{\response}{réponse : \hrulefill\\\\}

\begin{document}

\section*{Révisions}

% Pour chaques exercices, donnez ce qui sera inscrit pas la commande \codeintext{print}.

\begin{enumerate}

\item
\begin{lstlisting}
a = [1, 2, 3]
a = 3
a = "Python"
print(len(a))

\end{lstlisting}
\response % 6

\item
\begin{lstlisting}
a = [8, 2, 1]
a.sort()

print(a)
\end{lstlisting}
\response % [1, 2, 8]

\item
\begin{lstlisting}
a = "3"
b = 2 * a
print(b)
\end{lstlisting} 
\response % "bb"

\item
\begin{lstlisting}
a = 2
b = 9
c = a + b
print(b + c)
\end{lstlisting}
\response % 20

\item
\begin{lstlisting}
a = [1, 2, 3]
a.append(5)
b = min(a) + max(a)
print(b)
\end{lstlisting}
\response % 6
\pagebreak

\item
\begin{lstlisting}
def stupide(n):
	m = n + 1
	print (m)

stupide(6)

\end{lstlisting}
\response % 7

\item
\begin{lstlisting}
def weird(a, b, c):
	l = [a, b, c]
	l.reverse()
	return l

print(weird(5,6,8))
\end{lstlisting}
\response % [8, 6, 5]

\item
\begin{lstlisting}
def wtf(l):
	l.append(len(l)
	return l
	
l = [1, 2]
l = wtf(l)
l = wtf(l)
print(l)
\end{lstlisting}
\response % [1,2,2,3]

\pagebreak

\item
\begin{lstlisting}
import random

def nthng(n):
	a = 5
	n = n + a
	r =  random.randint(1,2)
	n = n -a
	return b
	
print(nthng(623))
	
\end{lstlisting}
\response % 623


\item
\begin{lstlisting}
l = [5, 2, 3]
a = 1
l = l[a:]
print(l)
\end{lstlisting}
\response % "The answer: 42"

\item
\begin{lstlisting}
a = [1]
b = 1
while(len(a) < 6 or a[-1] == 3)
	a.append(b)
	b = b + 1

\end{lstlisting}
\response % [1, 1, 2, 3]

\item
\begin{lstlisting}
def pluriel(mot, lettre="s"):
	mot = mot + lettre
	return mot
m = pluriel("pomme") + " et " + pluriel("chou", "x")
print(m)
	
\end{lstlisting}
\response % "pommes et choux"

\item
\begin{lstlisting}
def modulo(m, n):
	a = m // n
	return m - a * n
	
print(modulo(10, 3))
\end{lstlisting}
\response % 1

\item

\begin{lstlisting}

kame = "turtle"
c = ""
v = ['a', 'e', 'i', 'o', 'u', 'y']
for l in kame:
	if l not in v:
	c = c + l
	
print(l)

\end{lstlisting}
\response % [0, 1, 2]

\pagebreak

\item
\begin{lstlisting}

def is_p(a):
	if n // 2 == n / 2:
		return True
	return False

l = [1, 5, 8, 3, 10]
p = []
i = []

while len(l) > 0:
	n = l.pop(0)
	if is_p(n):
		p.append(n)
	else:
		i.append(n)
		
print(str(p) + " " str(i))
\end{lstlisting}
\response % [8, 10] [1, 5, 3]

%%% -----

\item
\begin{lstlisting}
a = [0, 2, 4, 2, -1, 3]
a = str(len(a))
a = a * 3
a = int(a)
a = a // 3 - 180
print(a) 
\end{lstlisting}
\response % 42

\item
\begin{lstlisting}
kuma = "bear"
akachan = kuma[:1]*3
print(akachan)

\end{lstlisting}
\response % bebe

\pagebreak
\item
\begin{lstlisting}
a = "Magic Dance"

title = ""

for a in l:
	a = a.upper()
	if a == 'A':
		for n in range(3):
			a = a + a
	title = title + a
	
print(title)
\end{lstlisting}
\response % "MAAAAGIC DAAAANCE"

\item
\begin{lstlisting}
a = [1, 2, 2, 4, 6]
b = a[1:3]
b[0] = 1
b.reverse()
b.insert(1, "X")
print(b) 
\end{lstlisting}
\response % [2, "X", 1]

\item
\begin{lstlisting}
def abracadadra(usagi):
	return usagi + usagi

rabbit = ["l", "4",  "p", "1", "n"]
print(abracadabra(rabbit))
\end{lstlisting}
\response %  ["l", "4",  "p", "1", "n", "l", "4",  "p", "1", "n"]

\end{enumerate}

\end{document}
