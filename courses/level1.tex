\chapter{Level 1}

%%<Mettre ici tout la partie introduction>

\section{Qu’est-ce qu’un langage de programmation?}

\epigraph{Un langage de programmation est une convention pour donner des ordres à un ordinateur. Ce n’est pas censé être obscur, bizarre et plein de pièges subtils.
Ca, ce sont les caractéristiques de la magie.}{\textit{Dave Small}}

Un langage de programmation est un moyen d’intéragir avec l’ordinateur afin de lui donner des instructions, le langage qui est \emph{intelligible}, 
sera \emph{interprété} afin d’être compris par la machine.

\begin{figure}[ht]
\centering
\includegraphics[scale=0.5]{translate_code.png} 
\end{figure}

\subsection{Stage 1}
 
Executer le script se trouvant dans le dossier \path{lv1/run\_game.py}

\section{Commander à l'ordinateur}

Pour ``commander'' l’ordinateur, on lui donne des \emph{instructions}, le plus souvent, c’est instructions seront regroupées dans un \emph{script}.

\subsection{Stage 2}

Dans l’exercice précédent, nous avons exécuté le script \path{run\_game.py}, ce script est remplie d’instructions, dont celle-ci:

\begin{lstlisting}
    dungeon_size = (5, 5)
    game = world.Game(dungeon_size)
    game.run()
\end{lstlisting}

Que ce passerait t’il si on modifiait l’instruction qui définit la grandeur du donjon dans le script?
Changez les données de dimension du donjon dans le script et observez ce qu'il se passe lors de l'exécution du script.

\subsection{\`{A} retenir}

Python est un \emph{langage sensible à la casse}\footnote{\textit{Case sensitive} en anglais.}, Ce qui veut dire qu'il fait la différence entre les majuscules et les minuscule. Autrement dit \codeintext{A} sera différent de \codeintext{a} et \codeintext{world} différent de \codeintext{World}

Donc si je remplace \codeintext{world} par \codeintext{World} dans le script précédent, il généra une erreur.

\begin{lstlisting}
    dungeon_size = (5, 5)
    game = World.Game(dungeon_size)
    game.run()
\end{lstlisting}

Vous devriez obtenir un message ressemblant à ``\codeintext{NameError: name 'World' is not defined}''

\section{Les instructions de sorties}

Il existe plusieurs sorte d’instruction, l’une d’elles sont les instructions de sortie.
Une instruction de sortie envoie vers une “sortie” se qu’on lui donne.
Une des sortie les plus couramment utilisé est la console et python, l’instruction de sortie vers la console est \codeintext{print}

Et donc voici le classique, mais indémodable “Hello World” en python

\begin{lstlisting}
    print("Hello World!")
\end{lstlisting}


\subsection{Stage 3}
Lorsque le donjon est créé, signalez-le par un message dans la \emph{console}.

\section{Les variables}

Souvent, il sera utile de stocké certaines valeur tout au long de l’exécution de votre code.
Par exemple pour stocker la valeur d’un calcul, où même afin de réutiliser ses valeurs plusieurs fois. Pour stocker des valeurs, on utilise des \emph{variables}.

La \emph{variables} est un moyen de stocker une valeur quelconque (un nombre, du texte, voir même des \emph{objets} plus complexes) dans la mémoire du programme.

Dans le cas de notre jeu, le donjon, ainsi que le héro sont contenu chacun dans une \emph{variable}.

\begin{lstlisting}
dungeon_size = (5, 5)
game = world.Game(dungeon_size)
hero = Game.hero
\end{lstlisting}

En python, l’\emph{affectation} d’une valeur se fait avec l’opérateur \emph{=}

Dans \codeintext{a=3}, On affecte \codeintext{3} à la variable \codeintext{a}\footnote{ En Python il suffit d’affecter une valeur à une variable pour qu’elle commence à exister, ce n’est pas vrai pour la plus part des langages (comme le C\# par exemple).}

\subsection{Stage 4}
Faites en sorte de stocker les dimensions du donjon dans des variables (par exemple \codeintext{longueur} et \codeintext{largeur}).

\section{Opération sur une chaîne de caractère}

Il vous sera parfois utile de savoir manipuler des chaînes de caractères, l'exemple classique est dans une ligne de dialogue, On voudrait dire au héro le nombre d'ennemis qui lui reste à tuer avant de finir la quête, or lorsque que vous codez, vous ne connaissez pas le nombre d'ennemis. Vous avez surement stocké cette information dans une variable et donc vous allez devoir intégrer cette variable à votre ligne de dialogue.

\subsection{Conversion}
Il y un moyen facile de convertir une variable en chaine de caractères\footnote{En python les chaine de caractère on le \emph{type} \codeintext{str} pour pour le mot anglais \textit{string}, qui veut dire chaîne}, il suffit d'utiliser la me fonction \codeintext{str}.

Si je voulais convertir, la variable \codeintext{nbr\_enemies} qui contient le chiffre \codeintext{3} je procéderais comme suit:

\begin{lstlisting}
	nbr_enemies = 3
	nbr_enemies = str(nbr_enemies)
\end{lstlisting}

A la fin de l'exécution de ce petit script, \codeintext{nbr\_enemies} ne vaut plus \codeintext{3} mais \codeintext{"3"}.

\subsection{Concaténation}
Le faite de coller deux chaînes de caractères l'une derrière l'autre porte un nom: \emph{la concaténation}.
En Python, l'opérateur pour \emph{concatener} deux variables ensemble est l'opérateur \codeintext{+}.

Par exemple, si je voulais concatener la variable contenant le nom du héro avec un message cela donnerai.
 
\begin{lstlisting}
	message = ", il te reste des ennemis à tuer."
	message = hero_name + message
\end{lstlisting}

Donc si le héro se nomme Brutor, la chaîne de caractère en fin de script sera égale à \codeintext{"Brutor, il te reste des ennemis à tuer."}

On peut même aller plus loin en mêlant la \emph{conversion} et la \emph{concaténation} en indiquant aussi dans le message le nombre d'ennemis qu'il reste.

\begin{lstlisting}
	message = ", il te reste " + str(nbr_enemies) + " ennemis à tuer."
	message = hero_name + message
\end{lstlisting}

\subsection{Stage 5}
Lorsque le donjon est créer, indiquer dans un message à la console la longueur et la largeur de celui-ci.

\section{Les fonctions et les méthodes, première approche}

Pour terminer ce premier level, on va mettre des mots sur deux choses que l'on a vu précédemment. Les \emph{fonction globales} et les \emph{méthodes}
Nous avons vu déjà vu plusieurs fonctions dans les parties précédentes:
\begin{itemize}
	\item[$\bullet$] \codeintext{print}
	\item[$\bullet$] \codeintext{str}
\end{itemize}

C'est fonction sont dîtes "globales". Ce type de fonction n'est pas liées à un type d'\emph{objet}.

Il existe aussi une série de fonction liées à des \emph{objets}, par exemple la méthode \emph{run} qui de la variable \emph{game}.
On appelle ses fonctions des \emph{méthodes} et elles sont liées à un type d'objets en particulié.

\subsection{Stage 6}
La variable du héro, sobrement appelée \codeintext{hero} est du type \codeintext{Hero}, c'est un type de variable que nous avons construit pour le jeu\footnote{Nous verrons plus tard qu'il est possible de définir ses propres type et de leur de donner les caractéristiques que l'on désire.}. Nous avons donné trois \emph{méthodes} à cette variable:

\begin{description}
	\item[\codeintext{turn\_left}:] ce qui fait tourner le héro à gauche.
	\item[\codeintext{turn\_right}:] ce qui fait tourner le héro à droite.
	\item[\codeintext{move}:] ce qui fait avancer le héro d'une "case".
\end{itemize}

Essayez de faire bouger le héro grâce à ces trois méthodes.

\section{Bonus Stage}
Pour compléter le stage bonus il faut que vous arriviez à ammener le héro sur la sortie du donjon. La sortie est représentée par la case bleu en haut à gauche de la pièce.
Le donjon doit au moins avoir une longueur et une largeur de 7.
